% mnras_template.tex
%
% LaTeX template for creating an MNRAS paper
%
% v3.0 released 14 May 2015
% (version numbers match those of mnras.cls)
%
% Copyright (C) Royal Astronomical Society 2015
% Authors:
% Keith T. Smith (Royal Astronomical Society)

% Change log
%
% v3.0 May 2015
%    Renamed to match the new package name
%    Version number matches mnras.cls
%    A few minor tweaks to wording
% v1.0 September 2013
%    Beta testing only - never publicly released
%    First version: a simple (ish) template for creating an MNRAS paper

%%%%%%%%%%%%%%%%%%%%%%%%%%%%%%%%%%%%%%%%%%%%%%%%%%
% Basic setup. Most papers should leave these options alone.
\documentclass[a4paper,fleqn,usenatbib]{mnras}

% MNRAS is set in Times font. If you don't have this installed (most LaTeX
% installations will be fine) or prefer the old Computer Modern fonts, comment
% out the following line
\usepackage{newtxtext,newtxmath}
% Depending on your LaTeX fonts installation, you might get better results with one of these:
%\usepackage{mathptmx}
%\usepackage{txfonts}

% Use vector fonts, so it zooms properly in on-screen viewing software
% Don't change these lines unless you know what you are doing
\usepackage[T1]{fontenc}
\usepackage{ae,aecompl}

\usepackage{mhchem}



\usepackage{graphicx}
\usepackage{subcaption}
\usepackage{float}
\usepackage{gensymb}
\usepackage{color}
\usepackage{booktabs,chemformula}
\usepackage[export]{adjustbox}

\usepackage{verbatim}
\usepackage{tabularx}

\usepackage{caption}
\usepackage{subcaption}
\usepackage{amsmath}

\usepackage{tikz}
\usepackage{hyperref}
\usepackage{longtable}
\usepackage{color}
\newcommand{\todo}[1]{\textcolor{red}{#1}}


\newcommand{\LamostGiants}{454180}
\newcommand{\project}[1]{\emph{#1}}
\newcommand{\lamost}{\project{LAMOST}}
\newcommand{\apogee}{\project{APOGEE}}

\newcommand{\tc}{\project{The Cannon}}

\newcommand{\teff}{T_{\rm eff}}
\newcommand{\logg}{\log_{10}[g\,({\rm cm\,s}^{-2})]}

%%%%%%%%%%%%%%%%%%% TITLE PAGE %%%%%%%%%%%%%%%%%%%

% Title of the paper, and the short title which is used in the headers.
% Keep the title short and informative.
\title[Mg-K anti-correlation in LAMOST]{Discovery of Mg-depleted and K-enhanced Stars in \textit{LAMOST}}

% The list of authors, and the short list which is used in the headers.
% If you need two or more lines of authors, add an extra line using \newauthor
\author[Kemp et al.]{
Alex J. Kemp,$^{1}$\thanks{E-mail: ajkem1@student.monash.edu}
Andrew R. Casey,$^{1,2}$
Mathew T. Miles,$^{1}$
Brodie J. Norfolk,$^{1}$\newauthor
John C. Lattanzio,$^{1}$
Amanda I. Karakas,$^{1}$
Kevin C. Schlaufman,$^{3}$\newauthor
Anna Y.Q Ho,$^{4}$
Melissa K. Ness,$^{5}$
David W. Hogg, $^{6}$
\\
% List of institutions
$^{1}$School of Physics \& Astronomy, Monash University, Clayton 3800, Victoria, Australia\\
$^{2}$ Faculty of Information Technology, Monash University, Clayton 3800, Victoria, Australia\\
$^{3}$, Department of Physics and Astronomy, Johns Hopkins University, Baltimore, MD 21218, USA\\
$^{4}$ Cahill Center for Astrophysics, California Institute of Technology, MC 249-17, 1200 E California Blvd, Pasadena, Ca, 91125, USA\\
$^{5}$ Max-Planck-Institut f$\ddot{u}$r Astronomie, K$\ddot{o}$nigsthul 17, D-69117 Heidelberg, Germany\\
$^{6}$ Center for Cosmology and Particle Physics, Department of Physics, New York University, 4 Washington Pl., room 424, New York, NY, 10003, USA\\
}

% These dates will be filled out by the publisher
\date{Accepted 2018 XX XX. Received 2018 YY YY; in original form 2018 ZZ ZZ}

% Enter the current year, for the copyright statements etc.
\pubyear{2018}


\begin{document}
\label{firstpage}
\pagerange{\pageref{firstpage}--\pageref{lastpage}}
\maketitle

% Abstract of the paper
\begin{abstract}

Stellar absorption spectra act as a fossil record of the conditions where the stars form. Stars with unusual elemental abundances are of particular interest; these rare stars often offer clues about rare astrophysical events or nucleosynthetic pathways. Stars with  depleted Mg and enhanced K abundances have thus far only been found in the two massive globular clusters NGC 2419 and NGC 2808. The origin of this abundance signature remains unknown, as does the reason for its apparent exclusivity to these two globular clusters. Here we present 113 candidate field stars identified from \LamostGiants \ giants observed by \lamost \ that have slightly depleted [Mg/Fe] and significantly enhanced [K/Fe] abundance ratios. These stars re present at a range of metallicities, and none have abundances that are as extreme as those in NGC 2419. These candidate stars should provide guidance for follow up surveys and modelling attempts attempting to determine the origin of the Mg-K anti-correlation.



\end{abstract}

% Select between one and six entries from the list of approved keywords.
% Don't make up new ones.
\begin{keywords}
methods: data analysis -- catalogues -- stars: chemically peculiar -- Galaxy: abundances -- Galaxy: evolution -- galaxies: globular clusters: individual: evolution
\end{keywords}

%%%%%%%%%%%%%%%%%%%%%%%%%%%%%%%%%%%%%%%%%%%%%%%%%%

%%%%%%%%%%%%%%%%% BODY OF PAPER %%%%%%%%%%%%%%%%%%

\section{Introduction}
\label{sec:intro}
NGC 2419 is the Milky Way's third most massive globular cluster, and its chemical composition makes it perhaps the most unusual star cluster in the Galaxy. Recent spectroscopic studies of red giant branch (RGB) stars in NGC 2419 revealed a strong anti-correlation between Mg and K abundances in nearly half of the studied stars, and weaker abundance relations in Si, Sc, Ca, Ti and V \citep{mucciarelli2012,cohenkirby2012}.

The only known mechanism for Mg depletion is the introduction of $\alpha$-poor and Fe-rich material released from Type Ia supernovae \citep{tsujimoto2012first} which locally pollutes the `normal' [Mg/Fe] ratios that arise from Type II supernovae. However, the two populations in NGC 2419 are indistinguishable in their [Fe/H] abundance \citep{cohenkirby2012}. Furthermore, Type Ia supernovae enrichment cannot account for the enhanced K abundance observed in NGC 2419. In fact, K is hard to make in sufficient quantities anywhere without invoking unrealistic temperature conditions or increasing reaction rates from measured values by several orders of magnitude. 


%Recently \cite{youngwooklee} used a Ca filter to identify two population in NGC 2419 corresponding to two generations of stars: G1, first generation metal poor stars, and G2, second generation stars displaying enhanced Ca and significantly increased He abundance ($\Delta Y = 0.19$). The Mg depleted population identified previously was found to be contained with the G2 population.

A targeted search looking for [K/Fe] in RGB and main sequence turn-off stars from clusters NGC 6752, NGC 6121, NGC 1904, 47 Tuc, NGC 6397, NGC 7099, and $\omega$ Centauri as well as 21 field stars found in all cases  that theK abundance fell within the bounds of the Mg-normal population in NGC 2419 \citep{carretta2013}. NGC 2808 is the only cluster other than NGC 2419 where an anticorrelation between Mg and K has been observed. All 4 of NGC 2808's known Mg depleted stars have been found to have an anti-correlation with K \citep{mucciarelli2015}, although the amplitude of these anomalous abundances was far weaker than that in NGC 2419. The fact that the Mg-K anti-correlation is apparently confined to these two globular clusters would seem to imply either a small population of unusual polluter stars or a single extremely massive polluter star.

An early modelling attempt at replicating the anomalous Mg/K abundances assumed hot-bottom burning in AGB and super-AGB (SAGB) stars \citep{ventura2012}. This study succeeded in reproducing the Mg-K anti-correlation through the nuclear reaction pathway \ce{^{36}Ar(p,\gamma)^{37}K(\beta ^+ \nu)^{37}Ar(e^-,\nu)^{37}Cl(p,\gamma)^{38}Ar(p,\gamma)^39K} (corrected as per \cite{iliadis2016}) for the production of K. However, abundance patterns consistent with NGC 2419 were only possible if the reaction cross section was set to 100 times the commonly accepted value or if the temperature at the base of the envelope could be made to exceed $\sim$ 150 MK.

A more recent attempt to replicate the Mg/K abundance signature in NGC 2419 by \cite{iliadis2016} also targetted abundances of Si, Sc, Ca, Ti and V, elements reported as having weak correlations with Mg by \cite{cohenkirby2012}. The ultimate goal was to constrain the temperatures and densities required to produce the chemical signature, and thereby provide insight into potential polluters. It is perhaps noteworthy that they identified a different main reaction pathway for K nucleosynthesis to that of \cite{ventura2012}, which is relegated to a minor secondary pathway. The main pathway they identify is \ce{^{36}Ar(p,\gamma)^{37}K(\beta ^+ \nu)^{37}Ar(p,\gamma)^{38}K(\beta ^+ \nu)^{38}Ar(p,\gamma)^39K}. 

It was found that (excepting the case of extremely high densities > $10^8$ g/cm$^3$) temperatures between 100 and 200 MK and densities between 10$^{-4}$ and 10$^8$ g/cm$^3$ were required. Using this information, core -- and shell --  burning of low-mass, high-mass and super-massive stars were ruled out as potential polluters, as well as regular AGB stars. Super-AGB (SAGB) stars, however, were regarded as potential candidates, with only a relatively small (roughly 10-20 MK) increase in temperature required to fall in the acceptable band of parameter space identified. The other potential candidate identified was novae, although the lack of detailed models of white dwarf accretion of metal-poor material limits certainty in this case. However, based on current novae frequency in globular clusters determined by \cite{kato2013novae}, \cite{iliadis2016} conclude that the amount of material that would be produced by novae is at most 1 \% of the total required mass to pollute 30 \% of NGC 2419.

Another proposed  polluter candidate is a pair-instability super nova \citep[PISN;][]{carretta2013}. Unique to extremely massive Population III stars, these events involve the total destruction of the star, with no black hole remnant left behind. The main argument for this idea is that the extreme rarity of these events, coupled with the huge masses of processed material released, could allow for the abundance signature in NGC 2419 to be the result of a single, extremely rare event, which would explain why it is not seen in any other globular clusters. However, the signature odd-even proton number abundance pattern associated with PISNs is absent from NGC 2419 \citep{cohenkirby2012}, making this particular scenario unlikely.

%It has been suggested that NGC 2419's position and size, being both very massive and very distant from the Milky Way compared to many other globular clusters, aids it in retaining enriched material from explosive events \citep{mucciarelli2012}. It is also perhaps noteworthy that NGC 2808 is also one of the more massive globular clusters of the Milky Way, although it is far closer to the Galactic Centre. But if this unknown pollution event is common among globular clusters and the cause of its apparent exclusivity to NGC 2808 and NGC 2419 is enhanced ejecta retention due to high mass, then other massive clusters such as $\omega$ Centauri would also be expected contain the signature, which to date has not been observed.

%While it is plausible that the increased mass and greater distance from the gaseous disk of the Milky Way would cause a clusters first generation polluter stars to have an increased effect on the second generation of stars, this effect would apply to all ejecta (assuming comparable energies). So while it might be expected that second generation stars in clusters similar to NGC 2419 made up of more material processed by the first generation than other lighter globular clusters, it doesn't explain why the ejecta from the first generation was apparently dominated by this as yet unknown polluter object resulting in the Mg-K anti-correlation. In the opinion of the author, it is the identity and physical characterisation of this unknown polluter that is of the greatest scientific interest, with such a result also hopefully answering the question of the signatures uniqueness.

Beyond the intrinsic scientific value of identifying the responsible mechanism for the puzzling Mg-K anti-correlation, a satisfactory explanation for the Mg-K signature could offer insight into the underestimation of K abundances in the Milky Way predicted by galactic chemical evolution models \citep{kobayashi2011}. Such an explanation could also be beneficial to efforts to understand globular cluster formation and evolution.

Here we use \lamost\ data to conduct the largest ever search for field stars that are enhanced in potassium and depleted in magnesium. The discovery (or non-discovery) of such stars will guide models that attempt to explain the Mg-K anti-correlation and related abundance phenomena. In Section \ref{sec:method} we outline our methods to identify candidates enhanced in K and depleted in Mg, and the follow-up observations and abundance analysis of those candidates. In Section \ref{sec:discussion} the results and their implications on possible poluters are discussed. Section \ref{sec:conclusion} summarises the main results.


%This study, which queries \LamostGiants\ \lamost\ giants for depleted Mg and enhanced K, seeks to provide additional information which can be used to guide and test models attempting to explain the Mg-K anti-correlation and related phenomena. The large sample size provides greater insight into the supposed exclusivity of the signature, and the stars identified present an important opportunity for follow up observations.


\section{Method \& Observations}
\label{sec:method}
The data set of \LamostGiants\ \lamost\ giants (from \lamost\ DR2) was prepared by \citet{ho2017} for use with the supervised machine learning code \tc. The \lamost\ data was shifted to rest frame and pseudo-continuum-normalised before it was resampled to a common uniform wavelength grid between 3905\,\AA\ and 9000\,\AA. \tc\ \citep{ness2016,ho2017} was used to infer effective temperature ($\teff$), surface gravity ($\logg$), metallicity ([Fe/H]), and alpha abundance relative to iron ([$\alpha$/Fe]) using 9,952 stars in common between \lamost\ and the higher spectral resolution \apogee\ \citep{alam2015} survey through the use of a data-driven spectral model. Unless otherwise stated, these transferred labels are those referred to throughout this study. For details regarding the preparatory work, model generation, and label transfer between the \apogee\ and \lamost\ surveys, the reader is directed to \citet{ho2017}. 

We identified potential Mg-K anti-correlated stars by searching for significant deviations in flux residuals. The flux residuals were taken as the normalised \lamost\ flux and the best-fitting data-driven model from \tc\ (where $f_{\textrm{residual}} = f_{\textrm{data}} - f_{\textrm{model}}$). A positive residual implies a higher observed normalised flux than expected by the model (less stellar absorption than predicted by the model), while a negative residual implies a lower observed normalised flux than expected (more stellar absorption than predicted by the model). 

We fitted a gaussian profile to the flux residuals for all three absorption lines in the Mg triplet (5167 \AA, 5172 \AA, 5184 \AA) as well as the K doublet (7665 \AA, 7699 \AA) for all \LamostGiants\ giants in \lamost\ data release 2 (DR2). For every star we recorded the profile amplitudes, wavelengths, widths, as well as associated measurement uncertainties of these quantities. We identified candidates by requiring they match at least one of the following three quality filters:
\begin{enumerate}
\item We required the amplitude $A$ of the profile at the Mg 5184 \AA \ line to satisfy $A_{{\rm Mg} @ 5184} > 0.05$ and the amplitude of the profile at the K 7665 line to satisfy $A_{{\rm K}\,@\,7665} < -0.05$. Both amplitudes must also be measured at more than $3\sigma$ ($|A|/\sigma_{A} \geq 3$).
\item We required any two of the three Mg triplet lines to satisfy $A > 0.05$ and at least one K line to have $A < -0.05$, and for those amplitudes to have $|A|/\sigma_{A} \geq 3$.
\item We required any two of the three Mg triplet lines to have $A > 0$ and both K lines to have $A < 0$, and for the spectra to have a signal-to-noise ($S/N$) ratio of $S/N > 30$ in \lamost, and a reported $\chi_{r}^2 < 3$ from \tc.
\end{enumerate} 
 
These filters identified 384 unique stars to be vetted manually. We visually inspected every candidate (multiple times) and excluded any candidate that showed any evidence of being a false positive, including candidates that exhibited data reduction issues, apparent absorption that was narrower than the expected spectral resolution, as well as 75 stars that exhibited chromospheric emission at H$\alpha$ wavelengths. The distilled catalogue contains \todo{113} candidate stars enhanced in K and depleted in Mg.

Abundances were calculated from the \lamost spectra \todo{@andy add how we did that, why we couldn't get Mg, error estimates and limitations.}

High resolution spectra of three of the candidate stars selected based on observability (J075043.12+204658.0, J120032.60+024438.2 and J091825.49+172114.5) were obtained using Magellan/MIKE. \todo{@andy add relevant info on abundance calculations.}

The presence of Na enhancements were visually estimated from the sample of 113 candidate stars' flux spectra. Flux based estimates placed at most 40 of the 113 stars (35\%) as being likely Na enhanced targets. Na abundances calculated from the \lamost\ spectra later revealed 30\% of the target stars with Na above 0.5 dex. \todo{@andy add anything specific to Na abundance methods from LAMOST}.


\begin{figure}
	\includegraphics[width=\columnwidth]{posterchildof13.png}
    \caption{Sample spectra for one of the candidate stars.}
    \label{posterchild}
\end{figure}

\begin{figure}
	\includegraphics[width=\columnwidth]{histof113.png}
    \caption{Histogram showing the metalicity distribution for the candidate stars.}
    \label{mhist}
\end{figure}

\begin{figure}
	\includegraphics[width=\columnwidth]{KvsMgab.png}
    \caption{Comparison of [K/Fe] vs [$\alpha$/Fe] for the candidate stars with [K/Fe] vs [Mg/Fe] abundances for NGC 2419 and NGC 2808 \citep{cohenkirby2012, mucciarelli2012, mucciarelli2015}. The dotted line is arbitrary, and is present simply as a visual aide.}
    \label{KvsMg}
\end{figure}

\begin{figure}
	\includegraphics[width=\columnwidth]{KvsFe.png}
    \caption{[K/Fe] vs [Fe/H] for the candidate stars, overlaid with [K/Fe] vs [Fe/H] for NGC 2419 and NGC 2808 \citep{cohenkirby2012, mucciarelli2012, mucciarelli2015}.}
    \label{KvsFe}
\end{figure}

\section{Discussion}
\label{sec:discussion}

\subsection{Selection effects}
\label{sec:selectioneffects}
\cite{ho2017}'s sample of \lamost\ giants carries with it certain selection effects as a result of both the \textit{APOGEE} DR12 and \lamost\ samples. Metal poor stars with [Fe/H] $\lesssim-2$ are mostly absent from the sample as they were likely too dissimilar to the training stars for label transfer from \textit{APOGEE}. Unfortunately, the target selection process for \lamost\ is non-invertible, and so selection effects present in \lamost\ are not considered here beyond comparing the candidate stars' metallicity distribution with that of \cite{ho2017}.

The candidate stars are distributed across the giant branch, implying the abundances detected are not associated with any NLTE effects or stage in stellar evolution.

\subsection{Association with known globular clusters}
\label{sec:globclustasoc}
Our sample was cross-matched against \cite{harris1996}'s data base of globular clusters. None of the stars appear to be associated with NGC 2419 or NGC 2808. From the sample, J074807.22+261514.2 is the closest star in the sky to a globular cluster, at 2.54\degree\ separation from Ko 2 \citep{koposov2007}. This separation is large for association with such a small, distant cluster and the star's metallicity and apparent magnitude make it even less likely to be a member.

The spread of metallicity shown in Figure \ref{mhist} further supports the sample being made up of field stars. The distribution reflects the original \LamostGiants \ sample of giants from which the candidates were drawn.
No stars are present at [Fe/H]$\sim -2$ -- the approximate metallicity of NGC 2419 -- nor are there many stars with [Fe/H]$\sim -1$ -- the metallicity of NGC 2808. Thus, we conclude that the candidates are not particular to any specific metallicity, including metallicites associated with NGC 2419 and NGC 2808.


\subsection{Abundances}
\label{sec:abundances}
Figure \ref{KvsMg} shows [K/Fe]  for the 113 star sample plotted against $[\alpha$/Fe] and overlaid with [K/Fe] from NGC 2419 and NGC 2808 plotted against [Mg/Fe] \citep{cohenkirby2012, mucciarelli2012, mucciarelli2015}. [K/Fe] vs [Mg/Fe] for the three stars from the sample observed using Magellan/MIKE is also shown. The use of [$\alpha$/Fe] as a proxy for [Mg/Fe] is supported by the [Mg/Fe] measurements from the high-resolution spectra, these being similar to the [$\alpha$/Fe] abundances from \tc.

[K/Fe] for the candidate stars is generally lower than that of the Mg-depleted population in NGC 2419, but higher than that of the Mg-normal population. It is noteworthy that both of these populations identified in NGC 2808 have significantly lower levels of [K/Fe] than the candidate stars.
Levels of [$\alpha$/Fe] fall within the lower limits of the Mg-normal population of both clusters, although as this represents only an upper limit the exact level of Mg depletion for the candidates is unclear. The [Mg/Fe] abundances derived from Magellan/MIKE's high resolution spectra suggest that at most a small depletion relative to the Mg-normal population in NGC 2419 may be present.

A similar relationship emerges when comparing with typical galactic numbers for [K/Fe] and [Mg/Fe], taken from \cite{kobayashi2011}.
Only the sample's [K/Fe] appears definitively anomalous, with galactic values for [K/Fe] around $0.3 \pm 0.2$ dex, while the identified candidate stars have a spread in [K/Fe] from around 0.8-1.3. Note that the Magellan/MIKE abundance for candidate star J091825.49+172114.5 has [K/Fe] $=1.54$, confirming that at least some of the sample are likely to have extremely high levels of [K/Fe].

The candidate stars' [K/Fe] also exhibits an interesting anti-correlation with total metallicity, shown in Figure \ref{KvsFe}, which is absent from measurements in the Milky Way. This may be consistent with a (somewhat) constant level of [K/H] between each star, which would then cause [K/Fe] to fall as [Fe/H] increases. It is interesting that the Mg-poor population of NGC 2419 appears to fit this relation.

However, for reasons related to the use of both a data-driven model and a physical model for calculating abundances \todo{@andy add any elaboration on what the problem was, if necessary}, it is possible that this effect is an artefact of processing rather than a physical relation. Without further observations of the sample stars, we are unable to say with certainty whether this phenomena is real or not.

The candidate stars' levels of [$\alpha$/Fe] were generally consistent with galactic levels of [Mg/Fe], although its possible a slight depletion is present. No variation with metallicity is observed for [$\alpha$/Fe].

%In contrast with the marked departure from galactic values for [K/Fe], [$\alpha$/Fe] varies from around 0-0.3 dex in the \lamost \ sample, similar to galactic levels of [Mg/Fe], which vary with metallicity from approximately 0.4$\pm$0.2 for [Fe/H] $\sim$ -1.5 to 0.1 $\pm$0.2 for [Fe/H] $\sim$ -0.3 \citep{kobayashi2011}. No variation with metallicity is observed within the \lamost sample's [$\alpha$/Fe].

[Na/Fe] was also measured for the \lamost\ data. Two distinct populations were observed, with a substantial minority of approximately 30\% of the sample having [Na/Fe] > 0.5, significantly higher than typical galactic abundances of [Na/Fe] $\lesssim$ 0.3 \citep{kobayashi2011}. However, this behaviour is very likely due to the effect of interstellar dust; it was found that the values of E(B-V), taken from IRSA's all-sky dust map \citep{schlafly2011}, were higher for the stars supposedly enriched in Na than for those displaying galactic levels. It worth mentioning briefly that the modelling work of \cite{prantzos2017} predicted an anti-correlation between Na and K in a 180 MK hydrogen burning environment appropriate to NGC 2808, while observational data for NGC 2808 implies a correlation between Na and K. For these newly identified candidate stars, neither relation between [K/Fe] and [Na/Fe] beyond probable dust effects is observed.

The implications of the calculated abundances are intriguing. The fact that from such a large sample of field stars none were found that replicated the extreme abundance pattern of NGC 2419 is strong evidence for the cluster's uniqueness regarding the extreme bi-modal nature of its abundance pattern.

The 113 star candidate pool of K enhanced and potentially Mg depleted stars represent a unique collection of stars; while not as K enhanced as the extreme population in NGC 2419, they exhibit K abundances well above typical galactic levels and a range of metalicities. It seems likely that whatever process is responsible for these anomalous field stars is similar to whatever caused the unusual stellar populations in NGC 2419 and NGC 2808.

\subsection{Impact on theories and scenarios regarding the Mg-K anti-correlation}

\label{sec:scenarios}
\begin{table*}
\centering
\begin{tabular}{llllllllll}
\hline
\textbf{2MASSID} & \textbf{RA} & \textbf{DEC} & \textbf{S/N} & \textbf{Vr (km/s)} & \textbf{Teff} & \textbf{Logg} & \textbf{[Fe/H]} & \textbf{[Alpha/Fe]} & {$\boldsymbol \chi^\textbf{2}$} \\ \hline
J060142.80+132321.4 & 90.43 & 13.39 & 45 & 33.28 & 4858 & 2.57 & -0.30 & 0.10 & 0.24 \\ \hline
J075043.12+204658.0 & 117.68 & 20.78 & 47 & 52.46 & 4795 & 2.53 & -0.47 & 0.06 & 0.40 \\ \hline
J032450.35+352401.8 & 51.21 & 35.40 & 143 & 95.63 & 4908 & 3.33 & -0.27 & 0.19 & 0.70 \\ \hline
J214721.25+023958.7 & 326.84 & 2.67 & 104 & -63.83 & 4649 & 2.22 & -0.68 & 0.26 & 0.81 \\ \hline
J040340.38+563902.4 & 60.92 & 56.65 & 30 & 39.87 & 4459 & 2.05 & -0.46 & 0.01 & 0.61 \\ \hline
J035455.85+594730.3 & 58.73 & 59.79 & 57 & -40.17 & 4956 & 2.81 & 0.08 & 0.04 & 0.56 \\ \hline
J035928.29+595932.8 & 59.87 & 59.99 & 35 & -38.07 & 4808 & 2.44 & -0.53 & 0.16 & 0.32 \\ \hline
J034458.82+592955.1 & 56.25 & 59.50 & 53 & -32.68 & 4811 & 2.66 & 0.06 & 0.10 & 0.64 \\ \hline
J220657.78+232138.3 & 331.74 & 23.36 & 50 & -27.58 & 4842 & 3.34 & -0.40 & 0.10 & 0.23 \\ \hline
J061323.14+325543.0 & 93.35 & 32.93 & 39 & -25.78 & 4861 & 2.51 & -0.40 & 0.16 & 0.42 \\ \hline
J005649.98+391722.9 & 14.21 & 39.29 & 41 & -46.77 & 4578 & 2.67 & 0.31 & 0.10 & 0.44 \\ \hline
J002619.36+565612.1 & 6.58 & 56.94 & 31 & -39.57 & 4205 & 1.49 & 0.22 & 0.06 & 0.76 \\ \hline
J040817.31+463252.5 & 62.07 & 46.55 & 92 & -1.80 & 5019 & 2.68 & -0.09 & 0.05 & 0.98 \\ \hline
J072156.18+225405.3 & 110.48 & 22.90 & 101 & -46.77 & 4840 & 2.54 & -0.33 & 0.17 & 0.69 \\ \hline
J040634.97+411354.9 & 61.65 & 41.23 & 45 & -35.38 & 4770 & 2.67 & -0.26 & 0.07 & 0.51 \\ \hline
J041608.62+433504.7 & 64.04 & 43.58 & 30 & -20.09 & 4967 & 2.90 & -0.17 & 0.04 & 0.33 \\ \hline
J013039.08+404843.8 & 22.66 & 40.81 & 71 & -80.34 & 4355 & 2.01 & -0.09 & 0.14 & 0.93 \\ \hline
J052924.25+494319.7 & 82.35 & 49.72 & 94 & -22.48 & 5087 & 3.12 & -0.35 & 0.11 & 0.70 \\ \hline
J061427.58+331544.3 & 93.61 & 33.26 & 51 & 3.30 & 4468 & 1.87 & -0.41 & 0.11 & 0.67 \\ \hline
J062128.63+343545.8 & 95.37 & 34.60 & 39 & 33.28 & 4831 & 2.58 & -0.23 & 0.08 & 0.27 \\ \hline
J030101.42+560042.3 & 45.26 & 56.01 & 88 & -39.27 & 4878 & 2.53 & -0.32 & 0.09 & 0.67 \\ \hline
J120032.60+024438.2 & 180.14 & 2.74 & 97 & 166.98 & 4633 & 2.33 & -0.93 & 0.26 & 0.89 \\ \hline
J163910.01+100545.3 & 249.79 & 10.10 & 39 & 123.81 & 4663 & 2.46 & -0.58 & 0.03 & 0.61 \\ \hline
\end{tabular}
\caption{Placeholder table}
\label{my-label}
\end{table*}


Significantly enhanced [K/Fe] is rare; significantly enhanced [K/Fe] without an accompanying depletion of [Mg/Fe] is unknown. The candidate stars are also spread across a wide range of metallicities.

This spread in metallicity is important: taken in isolation, it implies that the process responsible for this signature, rare as it is, is not necessarily tied to a certain epoch in the evolutionary history of our galaxy. Rather, these stars appear to be continuously forming in successive generations.

This interpretation argues against any theoretical hyper-massive Population III star, no matter how contrived, being the polluter candidate. While such an object might be somewhat appealing in the context of NGC 2419 given the uniqueness of the cluster, it cannot account for the production of K throughout the life of the galaxy.

As emphasized by \cite{prantzos2017}, a significant challenge presented by the signatures of NGC 2419 and NGC 2808 is obtaining both a depletion in Mg and an enhancement in K. Reducing Mg is possible through conversion of \ce{^{24}Mg} to \ce{^{23} Na} and through Mg-Al chains at temperatures around 75 MK --  attainable in hot-bottom burning environments in AGB stars -- while it takes temperatures upwards of 150 MK to produce K -- temperatures potentially developed at the bottom of the convective envelope in SAGB stars \citep{iliadis2016}. This implies that the Mg depletion and K enrichment likely occur at separate sites, assuming a relatively simple hydrogen burning system.

Allowing different sites for Mg depletion and K enrichment, one scenario that could possibly produce a large K enhanced signature along with (inconsistently) depleted Mg and enhanced Na throughout the lifetime of the Milky Way would be a binary in which one of the stars is a SAGB star and the other is some less massive star. The scenario might proceed as follows:

The SAGB star produces K (destroying Na in the process \citep{prantzos2017}) which is then mixed throughout its envelope. It then deposits its K rich outer layers onto its smaller companion through binary accretion. This smaller companion would continue to evolve long after the SAGB star has formed a white-dwarf remnant and will, depending on its mass after accretion, either undergo hot-bottom burning or, more likely, will not. If the companion star had sufficient mass to undergo hot bottom burning, then Na could be produced and Mg depleted \citep{prantzos2017}, with this material mixed throughout the envelope. In this scenario, it is observations of this smaller binary companion star that make up our sample of 113 giants.

Although this scenario allows for consistency with the total spread in metallicity, it doe not account for the uniqueness of the signal to NGC 2419 and NGC 2808 among globular clusters, nor does it satisfactorily explain why such a large proportion of stars in NGC 2419 exhibit the Mg-K anti-correlation. The idea that 30\% of the stars in NGC 2419 were once in a binary with SAGB stars is highly unlikely, to say the least.

A direct test of this scenario would be whether at least most of the candidates are binaries with O-Ne white dwarf remnants, which would indicate that they could have formed SAGB binaries once.

The anti-correlation between [K/Fe] and [Fe/H] could be interpreted as a relatively constant level of [K/H] present in at least one region of star forming gas, resulting in a steady decrease in [K/Fe] as [Fe/H] increases. This interpretation is somewhat supported by the consistency of the relation with levels of [K/Fe] in NGC 2419. If this relation is real, we can perhaps -- without saying anything about the polluter star -- use it to motivate an evolutionary sequence that can account for observations of [K/Fe] and [Mg/Fe] both in NGC 2419 and our new 113 field star sample.

Suppose some unknown event(s) unique to very metal poor (possibly zero-metallicity) environments occurs. This event results in a gas cloud depleted in Mg and enhanced in K. The generation of stars that form directly from this gas carries this signature, and displays a significant reduction in Mg abundance and enhancement in K abundance similar to NGC 2419's extreme population. We will refer to this generation of stars Generation 1.

When the most massive stars of Generation 1 die, they release metals into the gas cloud, including significant amounts of Mg, which is formed in copious amounts in core-collapse supernovae, but \textit{not} K, which is very difficult to form in significant quantities. The [Mg/Fe] depletion originally carried in the cloud is much harder to detect in the next generation of stars forming from this material, Generation 2, due to the influx of Mg released by Generation 1. The K enhancement however remains (in principle) relatively easy to detect, having only been reduced relative to [Fe/H] by the total amount of iron released. As this cycle continues, [K/Fe] would continue to lower proportionately with the rate of iron release, while depletions in [Mg/Fe] would quickly become impossible to detect.

This cycle allows for the Mg depletions to be preserved in globular clusters, where even the most massive clusters only form at most 2 generations of stars. But it also allows the Mg signal to effectively be lost within the disc of the galaxy, where many generations of stars may form. Thus it provides a way of linking the current observations of our 113 candidate stars with the observations of NGC 2419. Certainly the idea that both groups of anomalous stars arise from the same polluter source must be entertained.

However, beyond the question of whether the [K/Fe]-[Fe/H] anti-correlation is real or not, the extreme difficulty of forming a cloud of Mg depleted, K enriched material in the first place must be emphasized. Indeed, if we assume the primordial abundances (pre-mystery event) are similar to those of NGC 2419's Mg-normal population, then it is impossible. If the primordial cloud already has [Mg/Fe]$\approx 0.5$, to reduce this necessitates the addition of large amounts of iron rich material that is also very depleted in [Mg/Fe]. But as soon significant amounts of iron are added, [Fe/H] for the subsequent the Mg-depleted population should increase, while no significant [Fe/H] variation between the two populations is observed \citep{cohenkirby2012, mucciarelli2012}. This implies that the mystery polluter star may have released its material into an environment enriched only by Big-Bang Nucleosynthesis (BBNS).

More generally, the [K/Fe]-[Fe/H] anti-correlation observed is supportive of the origin of the Mg-K anti-correlation being temporally confined to an early epoch in the evolution of galaxies and globular clusters. However, this is not necessarily the only interpretation. For example, it could be that the K forming process is happening throughout the life of the galaxy, but the amount of K produced decreases with the metallicity of the polluter star, though there is no obvious reason this would be the case.



\section{Conclusions}
\label{sec:conclusion}
Of the 454180 giant sample from \lamost, 113 candidate stars were identified as likely to have enhanced K and depleted Mg based on their absorption spectra. These stars show every appearance of being field stars, nor do they have the hallmarks of being the product of NLTE effects. The candidate sample reflected the metallicity distribution of the base \textit{LAMOST} sample.

Upon conducting an abundance analysis, it was found that despite searching for both Mg depletions and K enhancements, the sample of stars showed no definitive Mg depletion, although it did display anomalously high levels of [K/Fe] relative to galactic levels. None of the candidates had calculated abundances of Mg or K consistent with the extreme populations of NGC 2419 or NGC 2808.

An intriguing anti-correlation between [K/Fe] and [Fe/H] is observed, although whether or not this relation is physical remains uncertain without further observations of the candidates.

\section*{Acknowledgements}
\todo{to be completed}

\begin{comment}
- LAMOST
- ARC DP (Casey)
- Magellan/MIKE and Australian government
- others???
\end{comment}
%%%%%%%%%%%%%%%%%%%%%%%%%%%%%%%%%%%%%%%%%%%%%%%%%%

%%%%%%%%%%%%%%%%%%%% REFERENCES %%%%%%%%%%%%%%%%%%

% The best way to enter references is to use BibTeX:

\bibliographystyle{mnras}
\bibliography{mgkbib} % if your bibtex file is called example.bib


\begin{comment}
\begin{table}[H]
\centering
\begin{tabular}{llllllllll}
\hline
\textbf{2MASSID} & \textbf{RA} & \textbf{DEC} & \textbf{S/N} & \textbf{Radial\_Velocity\_(km/s)} & \textbf{Teff} & \textbf{Logg} & \textbf{{[}Fe/H{]}} & \textbf{{[}Alpha/Fe{]}} & \textbf{$\chi^2$} \\ \hline
J234942.32+300654.0 & 357.43 & 30.11 & 20 & -195.46 & 4004 & 1.12 & -0.58 & 0.23 & 2.73 \\ \hline
J065334.23+131639.7 & 103.39 & 13.28 & 7 & 179.88 & 4896 & 2.61 & -0.71 & 0.02 & 0.18 \\ \hline
J112309.75+261255.9 & 170.79 & 26.22 & 3 & 113.02 & 4273 & 1.13 & -1.35 & 0.17 & 0.09 \\ \hline
J111651.71+253038.5 & 169.22 & 25.51 & 6 & 102.83 & 4658 & 2.90 & -0.50 & 0.02 & 0.08 \\ \hline
J045424.86+534628.5 & 73.60 & 53.77 & 6 & -36.57 & 4451 & 1.81 & -0.19 & 0.07 & 0.37 \\ \hline
J065857.80+081714.8 & 104.74 & 8.29 & 12 & 26.68 & 5070 & 3.30 & -0.61 & 0.10 & 0.18 \\ \hline
J045544.93+490023.3 & 73.94 & 49.01 & 6 & -38.07 & 4717 & 2.55 & -0.34 & 0.11 & 0.11 \\ \hline
J040739.62+423724.4 & 61.92 & 42.62 & 9 & -37.77 & 4075 & 1.54 & -0.32 & 0.08 & 0.28 \\ \hline
J043121.48+414757.2 & 67.84 & 41.80 & 8 & -12.59 & 4537 & 2.07 & -0.09 & 0.03 & 0.13 \\ \hline
J041900.21+422223.9 & 64.75 & 42.37 & 12 & -34.18 & 4263 & 1.77 & -0.55 & 0.09 & 0.28 \\ \hline
J041209.44+411801.0 & 63.04 & 41.30 & 35 & -20.69 & 4767 & 2.24 & -0.26 & 0.12 & 0.91 \\ \hline
J063648.82+130723.3 & 99.20 & 13.12 & 8 & 85.74 & 4925 & 2.93 & -0.57 & -0.03 & 0.19 \\ \hline
J070524.04+120351.0 & 106.35 & 12.06 & 19 & 60.26 & 4754 & 2.64 & -0.56 & -0.02 & 0.24 \\ \hline
J063501.25+134744.6 & 98.76 & 13.80 & 7 & 47.67 & 4612 & 2.33 & -0.40 & 0.01 & 0.19 \\ \hline
J053942.83+383727.8 & 84.93 & 38.62 & 18 & -85.74 & 4979 & 2.26 & -0.66 & 0.23 & 0.27 \\ \hline
J053123.26+421952.0 & 82.85 & 42.33 & 8 & 41.07 & 4785 & 2.59 & -0.26 & 0.01 & 0.26 \\ \hline
J052643.65+354130.3 & 81.68 & 35.69 & 9 & -18.89 & 4914 & 2.73 & -0.09 & 0.03 & 0.28 \\ \hline
J064618.82+050545.8 & 101.58 & 5.10 & 9 & 85.74 & 4772 & 2.66 & -0.49 & -0.01 & 0.28 \\ \hline
J063719.82+185602.6 & 99.33 & 18.93 & 5 & 15.29 & 5075 & 2.94 & -0.12 & 0.03 & 0.10 \\ \hline
J072912.52+073436.9 & 112.30 & 7.58 & 9 & 117.52 & 4835 & 3.18 & -0.44 & 0.33 & 0.34 \\ \hline
J193217.91+374505.6 & 293.07 & 37.75 & 28 & -70.15 & 4982 & 2.62 & -0.48 & 0.33 & 0.55 \\ \hline
J192937.24+390211.0 & 292.41 & 39.04 & 6 & -52.16 & 4819 & 3.00 & 0.37 & 0.12 & 0.21 \\ \hline
J010305.95+043445.9 & 15.77 & 4.58 & 90 & 20.69 & 4646 & 2.72 & 0.10 & 0.05 & 1.17 \\ \hline
J235454.61+110356.9 & 358.73 & 11.07 & 12 & -109.72 & 4794 & 3.07 & -0.16 & 0.30 & 0.21 \\ \hline
J000908.89+124821.9 & 2.29 & 12.81 & 10 & -78.55 & 4892 & 2.82 & -0.12 & 0.33 & 0.19 \\ \hline
J220115.66-001432.2 & 330.32 & -0.24 & 47 & 38.37 & 4925 & 3.30 & 0.06 & 0.23 & 0.32 \\ \hline
J064430.22+341606.3 & 101.13 & 34.27 & 9 & -40.17 & 4809 & 3.26 & 0.36 & 0.19 & 0.31 \\ \hline
J002907.61+354701.0 & 7.28 & 35.78 & 15 & -82.74 & 5116 & 3.56 & 0.24 & 0.17 & 0.18 \\ \hline
J033714.25+403015.1 & 54.31 & 40.50 & 34 & -119.32 & 4674 & 2.33 & 0.32 & 0.22 & 0.86 \\ \hline
J074807.22+261514.2 & 117.03 & 26.25 & 16 & -32.98 & 4846 & 2.69 & -0.23 & 0.05 & 0.16 \\ \hline
J032423.46+425429.6 & 51.10 & 42.91 & 23 & -56.74 & 4649 & 2.63 & -0.11 & -0.01 & 0.34 \\ \hline
J055043.44+162021.2 & 87.68 & 16.34 & 24 & 83.64 & 4587 & 2.49 & 0.21 & 0.07 & 0.28 \\ \hline
J055227.43+181628.0 & 88.11 & 18.27 & 28 & 8.09 & 5388 & 3.80 & -0.17 & 0.05 & 0.43 \\ \hline
J055935.08+150252.1 & 89.90 & 15.05 & 23 & 90.84 & 4487 & 2.15 & -0.68 & 0.01 & 4.35 \\ \hline
J063113.14+015900.3 & 97.80 & 1.98 & 14 & 78.55 & 4879 & 2.34 & -0.13 & 0.01 & 0.56 \\ \hline
J111550.58+104203.9 & 168.96 & 10.70 & 9 & 114.82 & 4519 & 1.72 & -1.41 & 0.32 & 0.17 \\ \hline
J111022.65+171251.0 & 167.59 & 17.21 & 12 & 247.33 & 4524 & 1.92 & -1.30 & 0.13 & 0.30 \\ \hline
J100054.32+185535.5 & 150.23 & 18.93 & 17 & 149.00 & 4373 & 2.30 & 0.01 & 0.01 & 0.39 \\ \hline
J055908.72+210037.7 & 89.79 & 21.01 & 6 & 55.76 & 4325 & 1.99 & -0.10 & -0.03 & 0.75 \\ \hline
J055704.30+215000.1 & 89.27 & 21.83 & 9 & 49.77 & 4691 & 2.94 & 0.37 & 0.00 & 0.30 \\ \hline
J104800.91+424311.0 & 162.00 & 42.72 & 6 & 163.99 & 5050 & 2.75 & -1.29 & 0.12 & 0.18 \\ \hline
J075129.58+024136.9 & 117.87 & 2.69 & 18 & 128.61 & 4741 & 2.39 & -0.61 & 0.05 & 0.28 \\ \hline
J064132.46+580400.9 & 100.39 & 58.07 & 24 & -11.69 & 5285 & 3.74 & -0.06 & 0.07 & 0.36 \\ \hline
J063908.57+381728.0 & 99.79 & 38.29 & 5 & 25.78 & 4853 & 2.73 & -0.52 & 0.02 & 0.17 \\ \hline
J102104.26+105617.4 & 155.27 & 10.94 & 17 & 171.48 & 4526 & 1.88 & -1.09 & 0.07 & 0.28 \\ \hline
J061228.04-040304.1 & 93.12 & -4.05 & 16 & -1.50 & 4928 & 3.24 & -0.56 & 0.09 & 0.19 \\ \hline
J061144.17-040845.1 & 92.93 & -4.15 & 23 & 84.54 & 4709 & 2.77 & -0.42 & -0.00 & 0.24 \\ \hline
J074848.57+210655.5 & 117.20 & 21.12 & 9 & 101.33 & 4447 & 2.25 & -0.52 & 0.04 & 0.11 \\ \hline
J092124.00+204559.5 & 140.35 & 20.77 & 12 & 125.31 & 4570 & 2.28 & -0.71 & 0.06 & 0.31 \\ \hline
J074146.65+145330.5 & 115.44 & 14.89 & 6 & 105.53 & 4701 & 2.00 & -0.71 & 0.04 & 0.18 \\ \hline
J073641.12+164710.6 & 114.17 & 16.79 & 6 & 90.84 & 4621 & 2.17 & -0.77 & 0.01 & 0.26 \\ \hline
J175211.80+273546.6 & 268.05 & 27.60 & 7 & -33.58 & 5489 & 3.71 & -0.03 & 0.14 & 0.15 \\ \hline
J194028.76+504047.6 & 295.12 & 50.68 & 22 & -61.76 & 4352 & 1.94 & -0.17 & 0.23 & 0.35 \\ \hline
J213914.06+025536.2 & 324.81 & 2.93 & 29 & 57.78 & 4488 & 2.40 & -0.37 & 0.05 & 0.46 \\ \hline
J011949.36+063411.4 & 19.96 & 6.57 & 479 & -28.18 & 4820 & 2.61 & -0.29 & 0.09 & 2.38 \\ \hline
J041055.45+570542.6 & 62.73 & 57.10 & 16 & -32.38 & 4141 & 1.66 & -0.23 & 0.07 & 0.47 \\ \hline
J035840.66+574934.5 & 59.67 & 57.83 & 17 & -56.06 & 4693 & 2.02 & -0.28 & 0.14 & 0.46 \\ \hline
J034552.01+595739.3 & 56.47 & 59.96 & 17 & -48.57 & 4627 & 2.07 & -0.38 & 0.16 & 0.41 \\ \hline
J193130.64+432204.2 & 292.88 & 43.37 & 16 & -92.64 & 4934 & 3.08 & -0.16 & 0.30 & 0.26 \\ \hline
J042753.62+365550.6 & 66.97 & 36.93 & 8 & -30.08 & 5249 & 2.91 & -0.21 & 0.18 & 0.30 \\ \hline
J061808.85+313338.2 & 94.54 & 31.56 & 12 & 42.57 & 4970 & 3.03 & -0.15 & 0.10 & 0.55 \\ \hline
J050740.59+492955.0 & 76.92 & 49.50 & 15 & 18.29 & 3879 & 0.96 & -0.72 & 0.06 & 0.54 \\ \hline
J050606.74+494925.0 & 76.53 & 49.82 & 25 & -28.78 & 4857 & 2.50 & -0.25 & 0.12 & 0.34 \\ \hline
J050551.93+493726.0 & 76.47 & 49.62 & 15 & -18.59 & 4961 & 2.85 & -0.47 & 0.07 & 0.27 \\ \hline
J051046.05+523848.3 & 77.69 & 52.65 & 47 & -46.17 & 4697 & 2.61 & -0.34 & 0.26 & 1.03 \\ \hline
J041321.95+520135.7 & 63.34 & 52.03 & 19 & -51.35 & 4694 & 2.48 & -0.05 & 0.05 & 0.50 \\ \hline
J065356.76+344513.1 & 103.49 & 34.75 & 21 & -23.38 & 4533 & 2.46 & -0.00 & 0.15 & 0.28 \\ \hline
J003451.10+424543.0 & 8.71 & 42.76 & 15 & -79.15 & 4460 & 2.10 & -0.56 & 0.24 & 0.31 \\ \hline
J040034.51+472103.8 & 60.14 & 47.35 & 18 & -55.82 & 4708 & 2.39 & -0.01 & 0.10 & 0.65 \\ \hline
J041304.73+470013.8 & 63.27 & 47.00 & 17 & 19.19 & 4290 & 1.79 & -0.37 & 0.03 & 0.79 \\ \hline
J035633.86+463749.2 & 59.14 & 46.63 & 17 & -54.93 & 4753 & 2.37 & -0.05 & 0.12 & 0.34 \\ \hline
J053627.38+581132.8 & 84.11 & 58.19 & 28 & -32.94 & 4457 & 2.52 & 0.28 & 0.05 & 0.49 \\ \hline
J052516.26+585912.5 & 81.32 & 58.99 & 19 & -75.07 & 4023 & 1.50 & -0.22 & 0.05 & 0.51 \\ \hline
J034826.25+350123.4 & 57.11 & 35.02 & 100 & -47.97 & 4710 & 2.89 & 0.04 & 0.38 & 4.72 \\ \hline
J035652.48+325937.2 & 59.22 & 32.99 & 24 & -20.99 & 4038 & 1.38 & -0.31 & 0.08 & 0.55 \\ \hline
J035345.63+340429.3 & 58.44 & 34.07 & 15 & -34.48 & 4830 & 3.23 & -0.56 & 0.06 & 0.14 \\ \hline
J042245.32+430359.8 & 65.69 & 43.07 & 16 & -35.08 & 4967 & 2.92 & -0.13 & 0.03 & 0.20 \\ \hline
J042241.36+414255.6 & 65.67 & 41.72 & 15 & -92.76 & 4975 & 3.32 & -0.66 & 0.02 & 0.27 \\ \hline
J053148.24+225045.6 & 82.95 & 22.85 & 15 & 11.69 & 3982 & 0.71 & -0.84 & 0.08 & 0.63 \\ \hline
J120844.42-012151.8 & 182.19 & -1.36 & 18 & 143.30 & 4783 & 2.28 & -0.84 & 0.28 & 0.48 \\ \hline
J091825.49+172114.5 & 139.61 & 17.35 & 11 & 232.04 & 4898 & 2.71 & -1.19 & 0.29 & 0.14 \\ \hline
J123413.43+160353.7 & 188.56 & 16.06 & 10 & 316.88 & 4734 & 2.63 & -0.99 & 0.03 & 0.17 \\ \hline
J054656.93+442800.0 & 86.74 & 44.47 & 17 & -41.37 & 4758 & 2.51 & -0.27 & 0.09 & 0.15 \\ \hline
J053423.46+460026.9 & 83.60 & 46.01 & 21 & -40.77 & 4893 & 2.73 & -0.18 & 0.10 & 0.26 \\ \hline
J052814.58+511912.4 & 82.06 & 51.32 & 16 & -64.76 & 4344 & 1.95 & -0.06 & 0.10 & 0.51 \\ \hline
J044403.36+543203.6 & 71.01 & 54.53 & 19 & -71.05 & 4625 & 1.91 & 0.25 & 0.07 & 1.42 \\ \hline
J043503.82+552514.6 & 68.77 & 55.42 & 19 & 9.59 & 4704 & 2.48 & -0.11 & 0.04 & 0.60 \\ \hline
J043418.64+535250.2 & 68.58 & 53.88 & 17 & -141.15 & 4204 & 1.44 & -0.64 & 0.19 & 0.76 \\ \hline
J070937.38+201517.2 & 107.41 & 20.25 & 9 & 148.10 & 3770 & 0.19 & -1.24 & 0.04 & 0.46 \\ \hline
J064032.35+335207.5 & 100.13 & 33.87 & 8 & -34.48 & 4717 & 2.28 & -0.44 & 0.24 & 0.24 \\ \hline
\end{tabular}
\caption{Remaining candidates}
\label{Resttab}
\end{table}
\end{comment}

%%%%%%%%%%%%%%%%%%%%%%%%%%%%%%%%%%%%%%%%%%%%%%%%%%


% Don't change these lines
\bsp	% typesetting comment
\label{lastpage}
\end{document}

% End of mnras_template.tex